\documentclass[a4paper, 12pt]{article}  % добавить leqno в [] для нумерации слева

%%% Работа с русским языком
\usepackage{cmap}
\usepackage[utf8]{inputenc}
\usepackage[english, russian]{babel}
\usepackage{pscyr}

%%% Дополнительная работа с математикой
\usepackage{amsmath,amsfonts,amssymb,amsthm,mathtools} % AMS
\usepackage{icomma} % "Умная" запятая

%% Номера формул
\mathtoolsset{showonlyrefs=true} % Показывать номера только у тех формул, на которые есть \eqref{} в тексте.

%% Шрифты
\usepackage{euscript} % Шрифт Евклид
\usepackage{mathrsfs} % Красивый матшрифт

%% Первый абзац после секшиона
\usepackage{indentfirst}

%% Свои команды
\DeclareMathOperator{\pphi}{\mathop{\varphi}}
\DeclareMathOperator{\eps}{\mathop{\varepsilon}}
\DeclareMathOperator{\conj}{\mathbb{\&}}

%% Перенос знаков в формулах (по Львовскому)
\newcommand*{\hm}[1]{#1\nobreak\discretionary{}
	{\hbox{$\mathsurround=0pt #1$}}{}}

%% Поля
\usepackage[left=1cm, right=1.4cm, top=1cm, bottom=1.3cm]{geometry}

%% Графика
\usepackage{tikz}
\usetikzlibrary{automata,positioning}

%% Переименовать список литературы
\addto\captionsrussian{\def\refname{Литература}}

%%% Заголовок
\author{Мат.клуб ``Тифаретник'' по С. Клини}
\title{Введение в метаматематику на троечку}
\date{\today}

%%% Эпиграф
\usepackage{epigraph}
\setlength\epigraphwidth{9cm}

%%% Работа с картинками
\usepackage{graphicx}  % Для вставки рисунков
\graphicspath{{images/}{images2/}}  % папки с картинками
\setlength\fboxsep{3pt} % Отступ рамки \fbox{} от рисунка
\setlength\fboxrule{1pt} % Толщина линий рамки \fbox{}
\usepackage{wrapfig} % Обтекание рисунков и таблиц текстом

%%% Добавляем код. Окружение lstlisting
\usepackage{listings}
\usepackage{color}
\definecolor{mygreen}{rgb}{0,0.6,0}
\definecolor{bggray}{RGB}{244,244,244}
\definecolor{mygray}{rgb}{0.5,0.5,0.5}
\definecolor{mymauve}{rgb}{0.58,0,0.82}
\lstset
{
	backgroundcolor=\color{bggray},   % choose the background color; you must add \usepackage{color} or \usepackage{xcolor}; should come as last argument
	basicstyle=\footnotesize,        % the size of the fonts that are used for the code
	breakatwhitespace=false,         % sets if automatic breaks should only happen at whitespace
	breaklines=true,                 % sets automatic line breaking
	captionpos=b,                    % sets the caption-position to bottom
	commentstyle=\color{mygreen},    % comment style
	deletekeywords={...},            % if you want to delete keywords from the given language
	escapeinside={\%*}{*)},          % if you want to add LaTeX within your code
	extendedchars=true,              % lets you use non-ASCII characters; for 8-bits encodings only, does not work with UTF-8
	%	firstnumber=1000,                % start line enumeration with line 1000
	%	frame=single,	                   % adds a frame around the code
	keepspaces=true,                 % keeps spaces in text, useful for keeping indentation of code (possibly needs columns=flexible)
	keywordstyle=\color{blue},       % keyword style
	language=c,                 	 % the language of the code
	morekeywords={*,...},            % if you want to add more keywords to the set
	numbers=none,                    % where to put the line-numbers; possible values are (none, left, right)
	numbersep=5pt,                   % how far the line-numbers are from the code
	numberstyle=\tiny\color{mygray}, % the style that is used for the line-numbers
	rulecolor=\color{black},         % if not set, the frame-color may be changed on line-breaks within not-black text (e.g. comments (green here))
	showspaces=false,                % show spaces everywhere adding particular underscores; it overrides 'showstringspaces'
	showstringspaces=false,          % underline spaces within strings only
	showtabs=false,                  % show tabs within strings adding particular underscores
	stepnumber=2,                    % the step between two line-numbers. If it's 1, each line will be numbered
	stringstyle=\color{mymauve},     % string literal style
	tabsize=3,	                     % sets default tabsize to 2 spaces
	title=\lstname                   % show the filename of files included with \lstinputlisting; also try caption instead of title	
}

%%% Ссылки
\usepackage{hyperref}

%%% Окружения
\theoremstyle{definition}
\newtheorem{theorem}{Теорема}
\newtheorem*{definition}{Определение}
\newtheorem*{pr}{Доказательство}

%%% Цифры в кружке
\newcommand*\circled[1]{\tikz[baseline=(char.base)]{
		\node[shape=circle,draw,inner sep=2pt] (char) {#1};}}

%%% Перечисление кастомное
\usepackage{enumitem}

%%% Естественный вывод
\usepackage{bussproofs}

%%% Несколько колонок для перечислений
\usepackage{multicol}

\usepackage{array}

%% no page numbering
\pagenumbering{gobble}

\begin{document}
	
	\subsection*{Дополнение неограниченного языка Дика как группа}
	
		Алфавит для слов $\mathscr{U} = \{ (_1, )_1, (_2, )_2, \dots, (_n, )_n \}$. Пусть $B_i$ это
		скобка типа $i$, a $B_i^{\backprime}$ другая того же типа. Будем порождать
		$\mathscr{U}^{\ast}$ конкатенацией ``$\cdot$'' с дополнительным условием:
			$$B_i  \cdot  B_i^{\backprime} = E$$
			
		И для двух комбинаций скобок:
			$$ B_{i_1}B_{i_2}\dots B_{i_n}B_i  \cdot  B_i^{\backprime}B_{j_1}B_{j_2}\dots B_{j_n} =
			B_{i_1}B_{i_2}\dots B_{i_n} \cdot E \cdot  B_{j_1}B_{j_2}\dots B_{j_n} $$
					
		Покажем, что для любой комбинации скобок существует обратная комбинация. Пусть слово 
		$A = B_{i_1}B_{i_2}\dots B_{i_{n-1}}B_{i_n}$. Тогда, чтобы получить обратный элемент,
		достаточно взять зеркальный образ $\overline{A}=B_{i_n}B_{i_{n-1}}\dots B_{i_2}B_{i_1}$ и
		заменить все скобки на другие того же типа 
		$\overline{A}^{\backprime}=B_{i_n}^{\backprime}B_{i_{n-1}}^{\backprime}\dots B_{i_2}^{\backprime}B_{i_1}^{\backprime}$:
		$$ A \cdot \overline{A}^{\backprime} = B_{i_1}B_{i_2}\dots B_{i_{n-1}}B_{i_n} \cdot
		  B_{i_n}^{\backprime}B_{i_{n-1}}^{\backprime}\dots B_{i_2}^{\backprime}B_{i_1}^{\backprime} =
		  E = \overline{A}^{\backprime} \cdot A $$
		
		Тогда $\mathscr{U}^{\ast}$ будет \textit{группой} порождённой подмножеством
		$\mathscr{U}$, т.е. $\mathscr{U}^{\ast} = \langle \mathscr{U} \rangle $.
		
		\begin{center}
			\begin{tabular}{|c||c|c|c|c|c|c|c|c}
				\hline
				$\cdot$ & $(_1$ & $)_1$ & $(_2$ & $)_2$ & \dots & $(_n$ & $)_n$ & \dots \\
				\hline
				\hline
				$(_1$ & $(_1(_1$ & $E$ & $(_1(_2$ & $(_1)_2$ & \dots  & $(_1(_n$ & $(_1)_n$ & \dots  \\
				\hline
				$)_1$ & $E$ & $)_1)_1$ & $)_1(_2$ & $)_1)_2$ & \dots  & $)_1(_n$ & $)_1)_n$ & \dots \\
				\hline
				 $(_2$ & $(_2(_1$ & $(_2)_1$ & $(_2(_2$ & $E$ & \dots & $(_2(_n$ & $(_2)_n$ & \dots  \\
				\hline
				$)_2$ & $)_2(_1$ & $)_2)_1$ & $E$ & $)_2)_2$ & \dots  & $)_2(_n$ & $)_2)_n$ & \dots  \\
				\hline
				\dots & \dots & \dots & \dots & \dots & \dots & \dots & \dots & \dots \\
				\hline
				$(_n$ & $(_n(_1$ & $(_n)_1$ & $(_n(_2$ & $(_n)_2$ & \dots & $(_n(_n$ & $E$ &\dots\\
				\hline
				$)_n$ & $)_n(_1$ & $)_n)_1$ & $)_n(_2$ & $)_n)_2$ & \dots & $E$ & $)_n)_n$ & \dots \\		
				\hline
				\dots & \dots & \dots & \dots & \dots & \dots & \dots & \dots & \dots \\		
			\end{tabular}
		\end{center}
\end{document}